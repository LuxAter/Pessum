\documentclass{subfiles}

\begin{document}
\subsection{Logging}
\newpage
\subsubsection{InitializeLogging}
\begin{lstlisting}
  void pessum::logging::InitializeLogging(std::string outputfile, bool recordtime, bool dev);
\end{lstlisting}
\begin{enumerate}
	\item[\emph{outputfile}] File name to save output log to.
	\item[\emph{recordtime}] Determines if the time in hh:mm:ss is recored with each log entry.
	\item[\emph{dev}] Determines if the log is to run in development mode.
\end{enumerate}
This funciton creates a logfile of the name \emph{outputfile}, which will be used for all logging output. If \emph{recordtime} is \emph{true} the time is included in the log entry. If \emph{recordtime} is \emph{false} then the time is not included. If \emph{dev} is \emph{true} then all log entries are recored to the file. If \emph{dev} is \emph{false} then only FATAL, ERROR, and WARNING log entries are recored to the file. \textbf{Note:} This function must be called before any other logging funcitons to work.
\begin{lstlisting}
  //Create a log file with the name "log_output.log", that will include times of each entry, and runs in development mode
  pessum::logging::InitializeLogging("log_output.log", true, true);
\end{lstlisting}
\newpage
\subsubsection{TerminateLogging}
\begin{lstlisting}
  void pessum::logging::TerminateLogging();
\end{lstlisting}
This funcition closes and saves the log file. \textbf{Note:} After this funciton is called no other logging funciton will work until InitializeLogging is called again.
\begin{lstlisting}
  //Terminate logging, and save the log file
  pessum::logging::TerminateLogging();
\end{lstlisting}
\newpage
\subsubsection{AddLogLocation}
\begin{lstlisting}
  int pessum::logging::AddLogLocation(std::string locationstring);
\end{lstlisting}
\begin{enumerate}
	\item[\emph{locationstring}] Name of location that is to be saved for logging.
\end{enumerate}
This function adds a file location into memory for later use in logging, permiting just an abbreviation or location index of the file path to be used in the log call. The abbreviation is made of the first three letters of each folder/file seperated by a '/'.\\
\textbf{Returns:} A integer value that point to the stored log location.
\begin{lstlisting}
  //Save a file of "folder/File.cpp" to the logging location
  int locationindex = pessum::logging::AddLogLocation("folder/File.cpp");
  //locationindex can be used in log calls to reference this locaiton, or the abbreviation "fol/Fil" can be used
\end{lstlisting}
\newpage
\subsubsection{Log}
\begin{lstlisting}
  void pessum::logging::Log(std::string str);
  void pessum::logging::Log(std::string typestr, std::string logstr, std::string locationstring);
  void pessum::logging::Log(std::string typestr, std::string logstr, std::string locationstr, std::string funcitonstr);
\end{lstlisting}
\begin{enumerate}
	\item
\end{enumerate}
\begin{lstlisting}

\end{lstlisting}
\newpage
\subsubsection{LogTimeStamp}
\begin{lstlisting}
  void pessum::logging::LogTimeStamp(bool date);
\end{lstlisting}
\begin{enumerate}
	\item
\end{enumerate}
\begin{lstlisting}

\end{lstlisting}
\newpage
\subsubsection{GetLocation}
\begin{lstlisting}
  std::string pessum::logging::GetLocation(int index);
  std::stirng pessum::logging::GetLocation(std::string str);
\end{lstlisting}
\begin{enumerate}
	\item
\end{enumerate}
\begin{lstlisting}

\end{lstlisting}
\newpage
\subsubsection{RemoveCaps}
\begin{lstlisting}
  std::string pessum::logging::RemoveCaps(std::string str);
\end{lstlisting}
\begin{enumerate}
	\item
\end{enumerate}
\begin{lstlisting}

\end{lstlisting}
\newpage
\subsubsection{GetType}
\begin{lstlisting}
  std::string pessum::logging::GetType(std::string str);
\end{lstlisting}
\begin{enumerate}
	\item
\end{enumerate}
\begin{lstlisting}

\end{lstlisting}
\newpage

\end{document}
