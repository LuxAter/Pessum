\documentclass{subfiles}

\begin{document}
\subsection{Logging}
\newpage
\subsubsection{AddLogLocation}
\begin{lstlisting}
  int pessum::logging::AddLogLocation(std::string locationstring);
\end{lstlisting}
\begin{enumerate}
	\item[\emph{locationstring}] Name of location that is to be saved for logging.
\end{enumerate}
This function adds a file location into memory for later use in logging, permiting just an abbreviation or location index of the file path to be used in the log call. The abbreviation is made of the first three letters of each folder/file seperated by a '/'.\\
\textbf{Returns:} A integer value that point to the stored log location.
\begin{lstlisting}
  //Save a file of "folder/File.cpp" to the logging location
  int locationindex = pessum::logging::AddLogLocation("folder/File.cpp");
  //locationindex can be used in log calls to reference this locaiton, or the abbreviation "fol/Fil" can be used
\end{lstlisting}
\newpage
\subsubsection{GetLocation}
\begin{lstlisting}
  std::string pessum::logging::GetLocation(int index);
  std::stirng pessum::logging::GetLocation(std::string str);
\end{lstlisting}
\begin{enumerate}
	\item[\emph{index}] Index that is assosiated with the desired location.
	\item[\emph{str}] String that is either an abbreviation, or is the desired location.
\end{enumerate}
Searches through the saved log locations, either by index value, or by abbreviation. \\
\textbf{Returns:} The string saved for specified location, and empty string if the specified location does not exist.
\begin{lstlisting}
  //A location must be added first
  int locationindex = pessum::logging::AddLogLocation("folder/subfolder/file");
  std::string location = pessum::logging::GetLocation(locationindex);
  location = pessum::logging::GetLocation("fol/sub/fil");
  //Both GetLocation calls will return "folder/subfolder/file"
\end{lstlisting}
\newpage
\subsubsection{GetType}
\begin{lstlisting}
  std::string pessum::logging::GetType(std::string str);
\end{lstlisting}
\begin{enumerate}
	\item[\emph{str}] Log type string.
\end{enumerate}
Takes a log type string, and converts to a standardized log type. Also expands abbreviations of default log types such as F to FATAL, or E to ERROR, ect.\\
\textbf{Returns:} A string representation of the log type.
\begin{lstlisting}
  pessum::logging::GetType("F")
  //Will return "FATAL"
\end{lstlisting}
\newpage
\subsubsection{InitializeLogging}
\begin{lstlisting}
  void pessum::logging::InitializeLogging(std::string outputfile, bool recordtime, bool dev);
\end{lstlisting}
\begin{enumerate}
	\item[\emph{outputfile}] File name to save output log to.
	\item[\emph{recordtime}] Determines if the time in hh:mm:ss is recored with each log entry.
	\item[\emph{dev}] Determines if the log is to run in development mode.
\end{enumerate}
This funciton creates a logfile of the name \emph{outputfile}, which will be used for all logging output. If \emph{recordtime} is \emph{true} the time is included in the log entry. If \emph{recordtime} is \emph{false} then the time is not included. If \emph{dev} is \emph{true} then all log entries are recored to the file. If \emph{dev} is \emph{false} then only FATAL, ERROR, and WARNING log entries are recored to the file. \textbf{Note:} This function must be called before any other logging funcitons to work.
\begin{lstlisting}
  //Create a log file with the name "log_output.log", that will include times of each entry, and runs in development mode
  pessum::logging::InitializeLogging("log_output.log", true, true);
\end{lstlisting}
\newpage
\subsubsection{Log}
\begin{lstlisting}

\end{lstlisting}
\begin{enumerate}
	\item
\end{enumerate}
\begin{lstlisting}

\end{lstlisting}
\newpage
\subsubsection{LogTimeStamp}
\begin{lstlisting}
  void pessum::logging::LogTimeStamp(bool date);
\end{lstlisting}
\begin{enumerate}
	\item[\emph{date}] Determines if the data is included in the time stamp.
\end{enumerate}
Logs a special time log, which includes the date and time in the format of www mmm dd hh:mm:ss yyyy. Where w is the day of the week, m is the month, d is the day, h is the hour, m is the minute, s is the second, and y is the year. If \emph{date} is true, the the date is included, if \emph{date} is false, then the data is removed from the log.
\begin{lstlisting}
  //Log the current time including the date
  pessum::logging::LogTimeStamp(true);
  //example result will be a log containing "Mon Jan 01 01:12:15 2017"
\end{lstlisting}
\newpage
\subsubsection{RemoveCaps}
\begin{lstlisting}
  std::string pessum::logging::RemoveCaps(std::string str);
\end{lstlisting}
\begin{enumerate}
	\item[\emph{str}] String to be converted.
\end{enumerate}
Converts all capital letters from \emph{str} into lower case letters.\\
\textbf{Returns:} \emph{str} will all capital letters converted to lowercase.
\begin{lstlisting}
  pessum::logging::RemoveCaps("Hello World")
  //Will return "hello world"
\end{lstlisting}
\newpage
\subsubsection{TerminateLogging}
\begin{lstlisting}
  void pessum::logging::TerminateLogging();
\end{lstlisting}
This funcition closes and saves the log file. \textbf{Note:} After this funciton is called no other logging funciton will work until InitializeLogging is called again.
\begin{lstlisting}
  //Terminate logging, and save the log file
  pessum::logging::TerminateLogging();
\end{lstlisting}
\newpage
\end{document}
