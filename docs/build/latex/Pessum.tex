%% Generated by Sphinx.
\def\sphinxdocclass{report}
\documentclass[letterpaper,10pt,english]{sphinxmanual}
\ifdefined\pdfpxdimen
   \let\sphinxpxdimen\pdfpxdimen\else\newdimen\sphinxpxdimen
\fi \sphinxpxdimen=.75bp\relax

\usepackage[utf8]{inputenc}
\ifdefined\DeclareUnicodeCharacter
 \ifdefined\DeclareUnicodeCharacterAsOptional\else
  \DeclareUnicodeCharacter{00A0}{\nobreakspace}
\fi\fi
\usepackage{cmap}
\usepackage[T1]{fontenc}
\usepackage{amsmath,amssymb,amstext}
\usepackage{babel}
\usepackage{times}
\usepackage[Bjarne]{fncychap}
\usepackage[dontkeepoldnames]{sphinx}

\usepackage{geometry}

% Include hyperref last.
\usepackage{hyperref}
% Fix anchor placement for figures with captions.
\usepackage{hypcap}% it must be loaded after hyperref.
% Set up styles of URL: it should be placed after hyperref.
\urlstyle{same}
\addto\captionsenglish{\renewcommand{\contentsname}{Contents:}}

\addto\captionsenglish{\renewcommand{\figurename}{Fig.}}
\addto\captionsenglish{\renewcommand{\tablename}{Table}}
\addto\captionsenglish{\renewcommand{\literalblockname}{Listing}}

\addto\extrasenglish{\def\pageautorefname{page}}

\setcounter{tocdepth}{1}



\title{Pessum Documentation}
\date{May 18, 2017}
\release{2.0}
\author{Arden Rasmussen}
\newcommand{\sphinxlogo}{\vbox{}}
\renewcommand{\releasename}{Release}
\makeindex

\begin{document}

\maketitle
\sphinxtableofcontents
\phantomsection\label{\detokenize{index::doc}}


Pessum is a base library for backend of programs. It provides a simple system
for logging. These logs can be saved to a file, or handled in the program
imediatly ({\hyperref[\detokenize{log::doc}]{\sphinxcrossref{\DUrole{doc}{Logging}}}}). Pessum also provides a simple system fo saving and
loading basic variables from a external file ({\hyperref[\detokenize{data::doc}]{\sphinxcrossref{\DUrole{doc}{Data}}}}).


\chapter{Data}
\label{\detokenize{data::doc}}\label{\detokenize{data:pessum-documentation}}\label{\detokenize{data:data}}
The data handling aspect of Pessum is primarily used to save data that is
changed must be maintained external to the program, and when the program ends,
the data can be saved to a different file.


\section{Functions}
\label{\detokenize{data:functions}}

\subsection{Load}
\label{\detokenize{data:load}}\index{pessum::Load (C++ function)}

\begin{fulllineitems}
\phantomsection\label{\detokenize{data:_CPPv2N6pessum4LoadENSt6stringE}}%
\pysigstartmultiline
\pysiglinewithargsret{std::vector\textless{}{\hyperref[\detokenize{datapoint:_CPPv2N6pessum9DataPointE}]{\sphinxcrossref{DataPoint}}}\textgreater{} \sphinxcode{}\sphinxbfcode{Load}}{std::string \sphinxstyleemphasis{file}}{}%
\pysigstopmultiline~

\begin{savenotes}\sphinxattablestart
\centering
\begin{tabulary}{\linewidth}[t]{|T|T|}
\hline

\sphinxcode{file}
&
Path to file to read data from
\\
\hline
\end{tabulary}
\par
\sphinxattableend\end{savenotes}

Reads data from a specified file, and returns a vector of the data in
{\hyperref[\detokenize{datapoint:_CPPv2N6pessum9DataPointE}]{\sphinxcrossref{\sphinxcode{DataPoint}}}}. The data will be converted into any basic types that it
can be converted to (\sphinxcode{int}, \sphinxcode{double}, \sphinxcode{bool}, \sphinxcode{std::string}).

\sphinxstylestrong{Return:} Vector of {\hyperref[\detokenize{datapoint:_CPPv2N6pessum9DataPointE}]{\sphinxcrossref{\sphinxcode{DataPoint}}}} containing information read
from \sphinxcode{file}.

\end{fulllineitems}



\subsection{Save}
\label{\detokenize{data:save}}\index{pessum::Save (C++ function)}

\begin{fulllineitems}
\phantomsection\label{\detokenize{data:_CPPv2N6pessum4SaveENSt6stringENSt6vectorI9DataPointEE}}%
\pysigstartmultiline
\pysiglinewithargsret{void \sphinxcode{}\sphinxbfcode{Save}}{std::string \sphinxstyleemphasis{file}, std::vector\textless{}{\hyperref[\detokenize{datapoint:_CPPv2N6pessum9DataPointE}]{\sphinxcrossref{DataPoint}}}\textgreater{} \sphinxstyleemphasis{data}}{}%
\pysigstopmultiline~

\begin{savenotes}\sphinxattablestart
\centering
\begin{tabulary}{\linewidth}[t]{|T|T|}
\hline

\sphinxcode{file}
&
Path to file to save data to
\\
\hline
\sphinxcode{data}
&
Data to save to file
\\
\hline
\end{tabulary}
\par
\sphinxattableend\end{savenotes}

Saves information from \sphinxcode{data} to \sphinxcode{file}.

\end{fulllineitems}



\chapter{Data Point}
\label{\detokenize{datapoint::doc}}\label{\detokenize{datapoint:data-point}}
The DataPoint class is a simple class that can contain a value for \sphinxcode{int},
\sphinxcode{double}, \sphinxcode{std::string}, or \sphinxcode{bool}. This is used for when the type of
the value is not known to the program. The type can then be determined through
the use of \sphinxcode{type}.


\section{Enumerators}
\label{\detokenize{datapoint:enumerators}}

\subsection{PessumDataType}
\label{\detokenize{datapoint:pessumdatatype}}\index{pessum::PessumDataType (C++ enum)}

\begin{fulllineitems}
\phantomsection\label{\detokenize{datapoint:_CPPv2N6pessum14PessumDataTypeE}}%
\pysigstartmultiline
\pysigline{\sphinxstrong{enum }\sphinxcode{}\sphinxbfcode{PessumDataType}}%
\pysigstopmultiline
Used to define the type of a {\hyperref[\detokenize{datapoint:_CPPv2N6pessum9DataPointE}]{\sphinxcrossref{\sphinxcode{DataPoint}}}}.


\begin{savenotes}\sphinxattablestart
\centering
\begin{tabulary}{\linewidth}[t]{|T|T|}
\hline

PESSUM\_NONE
&
0
\\
\hline
PESSUM\_INT
&
1
\\
\hline
PESSUM\_DOUBLE
&
2
\\
\hline
PESSUM\_STR
&
3
\\
\hline
PESSUM\_BOOL
&
4
\\
\hline
\end{tabulary}
\par
\sphinxattableend\end{savenotes}

\end{fulllineitems}



\section{Classes}
\label{\detokenize{datapoint:classes}}

\subsection{DataPoint}
\label{\detokenize{datapoint:datapoint}}\index{pessum::DataPoint (C++ class)}

\begin{fulllineitems}
\phantomsection\label{\detokenize{datapoint:_CPPv2N6pessum9DataPointE}}%
\pysigstartmultiline
\pysigline{\sphinxstrong{class }\sphinxcode{}\sphinxbfcode{DataPoint}}%
\pysigstopmultiline
Class structure to contain data of one of several different base types. The
data in a {\hyperref[\detokenize{datapoint:_CPPv2N6pessum9DataPointE}]{\sphinxcrossref{\sphinxcode{DataPoint}}}} class can be \sphinxcode{int}, \sphinxcode{double},
\sphinxcode{std::string}, or \sphinxcode{bool}.

\begin{sphinxVerbatim}[commandchars=\\\{\}]
\PYG{k}{class} \PYG{n+nc}{DataPoint}\PYG{p}{\PYGZob{}}
 \PYG{k}{public}\PYG{o}{:}
  \PYG{n}{DataPoint}\PYG{p}{(}\PYG{p}{)}\PYG{p}{;}
  \PYG{n}{DataPoint}\PYG{p}{(}\PYG{k+kt}{int} \PYG{n}{value}\PYG{p}{)}\PYG{p}{;}
  \PYG{n}{DataPoint}\PYG{p}{(}\PYG{k+kt}{double} \PYG{n}{value}\PYG{p}{)}\PYG{p}{;}
  \PYG{n}{DataPoint}\PYG{p}{(}\PYG{n}{std}\PYG{o}{:}\PYG{o}{:}\PYG{n}{string} \PYG{n}{value}\PYG{p}{)}\PYG{p}{;}
  \PYG{n}{DataPoint}\PYG{p}{(}\PYG{k+kt}{bool} \PYG{n}{value}\PYG{p}{)}\PYG{p}{;}

  \PYG{k+kt}{int} \PYG{n}{int\PYGZus{}value}\PYG{p}{,} \PYG{n}{type}\PYG{p}{;}
  \PYG{k+kt}{double} \PYG{n}{double\PYGZus{}value}\PYG{p}{;}
  \PYG{n}{std}\PYG{o}{:}\PYG{o}{:}\PYG{n}{string} \PYG{n}{string\PYGZus{}value}\PYG{p}{;}
  \PYG{k+kt}{bool} \PYG{n}{bool\PYGZus{}value}\PYG{p}{;}
\PYG{p}{\PYGZcb{}}\PYG{p}{;}
\end{sphinxVerbatim}

\end{fulllineitems}



\subsection{Constructors}
\label{\detokenize{datapoint:constructors}}

\subsubsection{DataPoint::DataPoint{[}1/5{]}}
\label{\detokenize{datapoint:datapoint-datapoint-1-5}}\index{pessum::DataPoint::DataPoint (C++ function)}

\begin{fulllineitems}
\phantomsection\label{\detokenize{datapoint:_CPPv2N6pessum9DataPoint9DataPointEv}}%
\pysigstartmultiline
\pysiglinewithargsret{\sphinxcode{DataPoint::}\sphinxbfcode{DataPoint}}{}{}%
\pysigstopmultiline
Default constructor, sets all values to default, and \sphinxcode{type} to
\sphinxcode{PESSUM\_NONE}.

\end{fulllineitems}



\subsubsection{DataPoint::DataPoint{[}2/5{]}}
\label{\detokenize{datapoint:datapoint-datapoint-2-5}}\index{pessum::DataPoint::DataPoint (C++ function)}

\begin{fulllineitems}
\phantomsection\label{\detokenize{datapoint:_CPPv2N6pessum9DataPoint9DataPointEi}}%
\pysigstartmultiline
\pysiglinewithargsret{\sphinxcode{DataPoint::}\sphinxbfcode{DataPoint}}{int \sphinxstyleemphasis{v}}{}%
\pysigstopmultiline~

\begin{savenotes}\sphinxattablestart
\centering
\begin{tabulary}{\linewidth}[t]{|T|T|}
\hline

\sphinxcode{value}
&
Integer value to use as set value
\\
\hline
\end{tabulary}
\par
\sphinxattableend\end{savenotes}

Constructor that sets the \sphinxcode{type} to \sphinxcode{PESSUM\_INT}, and sets
\sphinxcode{int\_value} to \sphinxcode{value}.

\end{fulllineitems}



\subsubsection{DataPoint::DataPoint{[}3/5{]}}
\label{\detokenize{datapoint:datapoint-datapoint-3-5}}\index{pessum::DataPoint::DataPoint (C++ function)}

\begin{fulllineitems}
\phantomsection\label{\detokenize{datapoint:_CPPv2N6pessum9DataPoint9DataPointEd}}%
\pysigstartmultiline
\pysiglinewithargsret{\sphinxcode{DataPoint::}\sphinxbfcode{DataPoint}}{double \sphinxstyleemphasis{v}}{}%
\pysigstopmultiline~

\begin{savenotes}\sphinxattablestart
\centering
\begin{tabulary}{\linewidth}[t]{|T|T|}
\hline

\sphinxcode{value}
&
Double value to use as set value
\\
\hline
\end{tabulary}
\par
\sphinxattableend\end{savenotes}

Constructor that sets the \sphinxcode{type} to \sphinxcode{PESSUM\_DOUBLE}, and sets
\sphinxcode{double\_value} to \sphinxcode{value}.

\end{fulllineitems}



\subsubsection{DataPoint::DataPoint{[}4/5{]}}
\label{\detokenize{datapoint:datapoint-datapoint-4-5}}\index{pessum::DataPoint::DataPoint (C++ function)}

\begin{fulllineitems}
\phantomsection\label{\detokenize{datapoint:_CPPv2N6pessum9DataPoint9DataPointENSt6stringE}}%
\pysigstartmultiline
\pysiglinewithargsret{\sphinxcode{DataPoint::}\sphinxbfcode{DataPoint}}{std::string \sphinxstyleemphasis{v}}{}%
\pysigstopmultiline~

\begin{savenotes}\sphinxattablestart
\centering
\begin{tabulary}{\linewidth}[t]{|T|T|}
\hline

\sphinxcode{value}
&
String value to use as set value
\\
\hline
\end{tabulary}
\par
\sphinxattableend\end{savenotes}

Constructor that sets the \sphinxcode{type} to \sphinxcode{PESSUM\_STR}, and sets
\sphinxcode{string\_value} to \sphinxcode{value}.

\end{fulllineitems}



\subsubsection{DataPoint::DataPoint{[}5/5{]}}
\label{\detokenize{datapoint:datapoint-datapoint-5-5}}\index{pessum::DataPoint::DataPoint (C++ function)}

\begin{fulllineitems}
\phantomsection\label{\detokenize{datapoint:_CPPv2N6pessum9DataPoint9DataPointEb}}%
\pysigstartmultiline
\pysiglinewithargsret{\sphinxcode{DataPoint::}\sphinxbfcode{DataPoint}}{bool \sphinxstyleemphasis{v}}{}%
\pysigstopmultiline~

\begin{savenotes}\sphinxattablestart
\centering
\begin{tabulary}{\linewidth}[t]{|T|T|}
\hline

\sphinxcode{value}
&
Boolian value to use as set value
\\
\hline
\end{tabulary}
\par
\sphinxattableend\end{savenotes}

Constructor that sets the \sphinxcode{type} to \sphinxcode{PESSUM\_BOOL}, and sets
\sphinxcode{bool\_value} to \sphinxcode{value}.

\end{fulllineitems}



\section{Functions}
\label{\detokenize{datapoint:functions}}

\subsection{Make\_DataPoint}
\label{\detokenize{datapoint:make-datapoint}}\index{pessum::Make\_DataPoint (C++ function)}

\begin{fulllineitems}
\phantomsection\label{\detokenize{datapoint:_CPPv2N6pessum14Make_DataPointENSt6stringE}}%
\pysigstartmultiline
\pysiglinewithargsret{{\hyperref[\detokenize{datapoint:_CPPv2N6pessum9DataPointE}]{\sphinxcrossref{DataPoint}}} \sphinxcode{}\sphinxbfcode{Make\_DataPoint}}{std::string \sphinxstyleemphasis{str}}{}%
\pysigstopmultiline~

\begin{savenotes}\sphinxattablestart
\centering
\begin{tabulary}{\linewidth}[t]{|T|T|}
\hline

\sphinxcode{str}
&
String to convert to {\hyperref[\detokenize{datapoint:_CPPv2N6pessum9DataPointE}]{\sphinxcrossref{\sphinxcode{DataPoint}}}}
\\
\hline
\end{tabulary}
\par
\sphinxattableend\end{savenotes}

This function takes a string, and reads it. If the string can be converted
into some other type (\sphinxcode{int}, \sphinxcode{double}, or \sphinxcode{bool}), it is converted.
Then everything is saved into a {\hyperref[\detokenize{datapoint:_CPPv2N6pessum9DataPointE}]{\sphinxcrossref{\sphinxcode{DataPoint}}}}.

\sphinxstylestrong{Return:} {\hyperref[\detokenize{datapoint:_CPPv2N6pessum9DataPointE}]{\sphinxcrossref{\sphinxcode{DataPoint}}}} containing the reducd type of the string data.

\end{fulllineitems}



\chapter{Logging}
\label{\detokenize{log:logging}}\label{\detokenize{log::doc}}
The logging functionality of Pessum, is very simple. It permits logs entries to
be added to a set of global log entries for that occurance of the program. These
log entries can then be handled by provided functions when they are added, or
they can be retreaved later with one of several log retreval functions. The
entire list of log entries can also be saved to an external file for review
after program termination.


\section{Enumerators}
\label{\detokenize{log:enumerators}}

\subsection{LogType}
\label{\detokenize{log:logtype}}\index{pessum::LogType (C++ enum)}

\begin{fulllineitems}
\phantomsection\label{\detokenize{log:_CPPv2N6pessum7LogTypeE}}%
\pysigstartmultiline
\pysigline{\sphinxstrong{enum }\sphinxcode{}\sphinxbfcode{LogType}}%
\pysigstopmultiline
Used to define the type/importance of the log call.


\begin{savenotes}\sphinxattablestart
\centering
\begin{tabulary}{\linewidth}[t]{|T|T|}
\hline

ERROR
&
0
\\
\hline
WARNING
&
1
\\
\hline
TRACE
&
2
\\
\hline
DEBUG
&
3
\\
\hline
SUCCESS
&
4
\\
\hline
INFO
&
5
\\
\hline
DATA
&
6
\\
\hline
NONE
&
7
\\
\hline
\end{tabulary}
\par
\sphinxattableend\end{savenotes}

\end{fulllineitems}



\section{Functions}
\label{\detokenize{log:functions}}

\subsection{Log}
\label{\detokenize{log:log}}\index{pessum::Log (C++ function)}

\begin{fulllineitems}
\phantomsection\label{\detokenize{log:_CPPv2N6pessum3LogEiNSt6stringENSt6stringEz}}%
\pysigstartmultiline
\pysiglinewithargsret{void \sphinxcode{}\sphinxbfcode{Log}}{int \sphinxstyleemphasis{type}, std::string \sphinxstyleemphasis{msg}, std::string \sphinxstyleemphasis{func}, …}{}%
\pysigstopmultiline~

\begin{savenotes}\sphinxattablestart
\centering
\begin{tabulary}{\linewidth}[t]{|T|T|}
\hline

\sphinxcode{type}
&
Type of log entry from {\hyperref[\detokenize{log:_CPPv2N6pessum7LogTypeE}]{\sphinxcrossref{\sphinxcode{LogType}}}}
\\
\hline
\sphinxcode{msg}
&
Format string of log entry
\\
\hline
\sphinxcode{func}
&
The name of the function creating the log entry
\\
\hline
\sphinxcode{...}
&
Additional formating args for \sphinxcode{msg}
\\
\hline
\end{tabulary}
\par
\sphinxattableend\end{savenotes}

Core function for all logging output, \sphinxcode{msg} is a format string with
additional arguments as needed from \sphinxcode{...}. Formated string and log type
are saved to {\hyperref[\detokenize{log:_CPPv2N6pessum11global_logsE}]{\sphinxcrossref{\sphinxcode{global\_logs}}}}.

\end{fulllineitems}



\subsection{GetLog}
\label{\detokenize{log:getlog}}

\subsubsection{GetLog}
\label{\detokenize{log:id1}}\index{pessum::GetLog (C++ function)}

\begin{fulllineitems}
\phantomsection\label{\detokenize{log:_CPPv2N6pessum6GetLogEi}}%
\pysigstartmultiline
\pysiglinewithargsret{std::string \sphinxcode{}\sphinxbfcode{GetLog}}{int \sphinxstyleemphasis{type}}{}%
\pysigstopmultiline~

\begin{savenotes}\sphinxattablestart
\centering
\begin{tabulary}{\linewidth}[t]{|T|T|}
\hline

\sphinxcode{type}
&
The type of log entry to find and retrieve
\\
\hline
\end{tabulary}
\par
\sphinxattableend\end{savenotes}

Gets last log entry of specified type with formated string.

\sphinxstylestrong{Return:} Formated string of log entry.

\end{fulllineitems}



\subsubsection{FGetLog}
\label{\detokenize{log:fgetlog}}\index{pessum::FGetLog (C++ function)}

\begin{fulllineitems}
\phantomsection\label{\detokenize{log:_CPPv2N6pessum7FGetLogEi}}%
\pysigstartmultiline
\pysiglinewithargsret{std::pair\textless{}int, std::string\textgreater{} \sphinxcode{}\sphinxbfcode{FGetLog}}{int \sphinxstyleemphasis{type}}{}%
\pysigstopmultiline~

\begin{savenotes}\sphinxattablestart
\centering
\begin{tabulary}{\linewidth}[t]{|T|T|}
\hline

\sphinxcode{type}
&
The type of log entry to find and retrieve
\\
\hline
\end{tabulary}
\par
\sphinxattableend\end{savenotes}

Gets last log entry of specified type with log type and formated string.

\sphinxstylestrong{Return:} Pair of log type and formated string of log entry.

\end{fulllineitems}



\subsubsection{IGetLog}
\label{\detokenize{log:igetlog}}\index{pessum::IGetLog (C++ function)}

\begin{fulllineitems}
\phantomsection\label{\detokenize{log:_CPPv2N6pessum7IGetLogEi}}%
\pysigstartmultiline
\pysiglinewithargsret{std::string \sphinxcode{}\sphinxbfcode{IGetLog}}{int \sphinxstyleemphasis{index}}{}%
\pysigstopmultiline~

\begin{savenotes}\sphinxattablestart
\centering
\begin{tabulary}{\linewidth}[t]{|T|T|}
\hline

\sphinxcode{index}
&
The index of the log entry from {\hyperref[\detokenize{log:_CPPv2N6pessum11global_logsE}]{\sphinxcrossref{\sphinxcode{global\_logs}}}}
\\
\hline
\end{tabulary}
\par
\sphinxattableend\end{savenotes}

Gets log entry of specified index with formated string.

\sphinxstylestrong{Return:} Formated string of log entry.

\end{fulllineitems}



\subsubsection{IFGetLog}
\label{\detokenize{log:ifgetlog}}\index{pessum::IFGetLog (C++ function)}

\begin{fulllineitems}
\phantomsection\label{\detokenize{log:_CPPv2N6pessum8IFGetLogEi}}%
\pysigstartmultiline
\pysiglinewithargsret{std::string \sphinxcode{}\sphinxbfcode{IFGetLog}}{int \sphinxstyleemphasis{index}}{}%
\pysigstopmultiline~

\begin{savenotes}\sphinxattablestart
\centering
\begin{tabulary}{\linewidth}[t]{|T|T|}
\hline

\sphinxcode{index}
&
The index of the log entry from {\hyperref[\detokenize{log:_CPPv2N6pessum11global_logsE}]{\sphinxcrossref{\sphinxcode{global\_logs}}}}
\\
\hline
\end{tabulary}
\par
\sphinxattableend\end{savenotes}

Gets log entry of specified index with log type formated string.

\sphinxstylestrong{Return:} Pair of log type and formated string of log entry.

\end{fulllineitems}



\subsubsection{VGetLog}
\label{\detokenize{log:vgetlog}}\index{pessum::VGetLog (C++ function)}

\begin{fulllineitems}
\phantomsection\label{\detokenize{log:_CPPv2N6pessum7VGetLogEii}}%
\pysigstartmultiline
\pysiglinewithargsret{std::vector\textless{}std::string\textgreater{} \sphinxcode{}\sphinxbfcode{VGetLog}}{int \sphinxstyleemphasis{start}, int \sphinxstyleemphasis{end}}{}%
\pysigstopmultiline~

\begin{savenotes}\sphinxattablestart
\centering
\begin{tabulary}{\linewidth}[t]{|T|T|}
\hline

\sphinxcode{start}
&
The first index value from {\hyperref[\detokenize{log:_CPPv2N6pessum11global_logsE}]{\sphinxcrossref{\sphinxcode{global\_logs}}}}
\\
\hline
\sphinxcode{end}
&
The last index value from {\hyperref[\detokenize{log:_CPPv2N6pessum11global_logsE}]{\sphinxcrossref{\sphinxcode{global\_logs}}}}
\\
\hline
\end{tabulary}
\par
\sphinxattableend\end{savenotes}

Get a set of log entries between (inclusive) specified start and end index
with formated string.

\sphinxstylestrong{Return:} Vector of strings of log entries.

\end{fulllineitems}



\subsubsection{VFGetLog}
\label{\detokenize{log:vfgetlog}}\index{pessum::VFGetLog (C++ function)}

\begin{fulllineitems}
\phantomsection\label{\detokenize{log:_CPPv2N6pessum8VFGetLogEii}}%
\pysigstartmultiline
\pysiglinewithargsret{std::vector\textless{}std::string\textgreater{} \sphinxcode{}\sphinxbfcode{VFGetLog}}{int \sphinxstyleemphasis{start}, int \sphinxstyleemphasis{end}}{}%
\pysigstopmultiline~

\begin{savenotes}\sphinxattablestart
\centering
\begin{tabulary}{\linewidth}[t]{|T|T|}
\hline

\sphinxcode{start}
&
The first index value from {\hyperref[\detokenize{log:_CPPv2N6pessum11global_logsE}]{\sphinxcrossref{\sphinxcode{global\_logs}}}}
\\
\hline
\sphinxcode{end}
&
The last index value from {\hyperref[\detokenize{log:_CPPv2N6pessum11global_logsE}]{\sphinxcrossref{\sphinxcode{global\_logs}}}}
\\
\hline
\end{tabulary}
\par
\sphinxattableend\end{savenotes}

Get a set of log entries between (inclusive) specified start and end index
with log type and formated string.

\sphinxstylestrong{Return:} Vector of pairs of log type and formated stirng of log entry.

\end{fulllineitems}



\subsection{SetLogHandle}
\label{\detokenize{log:setloghandle}}

\subsubsection{SetLogHandle{[}1/2{]}}
\label{\detokenize{log:setloghandle-1-2}}\index{pessum::SetLogHandle (C++ function)}

\begin{fulllineitems}
\phantomsection\label{\detokenize{log:_CPPv2N6pessum12SetLogHandleEPFvNSt4pairIiNSt6stringEEEE}}%
\pysigstartmultiline
\pysiglinewithargsret{void \sphinxcode{}\sphinxbfcode{SetLogHandle}}{void (*\sphinxstyleemphasis{handle})}{std::pair\textless{}int, std::string\textgreater{}}{}{}%
\pysigstopmultiline~

\begin{savenotes}\sphinxattablestart
\centering
\begin{tabulary}{\linewidth}[t]{|T|T|}
\hline

\sphinxcode{handle}
&
Pointer to function with return of void and args of a pair of int and string
\\
\hline
\end{tabulary}
\par
\sphinxattableend\end{savenotes}

Sets {\hyperref[\detokenize{log:_CPPv2N6pessum15log_handle_fullE}]{\sphinxcrossref{\sphinxcode{log\_handle\_full}}}} to given pointer.

\end{fulllineitems}



\subsubsection{SetLogHandle{[}2/2{]}}
\label{\detokenize{log:setloghandle-2-2}}\index{pessum::SetLogHandle (C++ function)}

\begin{fulllineitems}
\phantomsection\label{\detokenize{log:_CPPv2N6pessum12SetLogHandleEPFvNSt6stringEE}}%
\pysigstartmultiline
\pysiglinewithargsret{void \sphinxcode{}\sphinxbfcode{SetLogHandle}}{void (*\sphinxstyleemphasis{handle})}{std::string}{}{}%
\pysigstopmultiline~

\begin{savenotes}\sphinxattablestart
\centering
\begin{tabulary}{\linewidth}[t]{|T|T|}
\hline

\sphinxcode{handle}
&
Pointer to function with return of void and args of a string
\\
\hline
\end{tabulary}
\par
\sphinxattableend\end{savenotes}

\end{fulllineitems}



\subsection{GetTypeStr}
\label{\detokenize{log:gettypestr}}\index{pessum::GetTypeStr (C++ function)}

\begin{fulllineitems}
\phantomsection\label{\detokenize{log:_CPPv2N6pessum10GetTypeStrEi}}%
\pysigstartmultiline
\pysiglinewithargsret{std::string \sphinxcode{}\sphinxbfcode{GetTypeStr}}{int \sphinxstyleemphasis{type}}{}%
\pysigstopmultiline~

\begin{savenotes}\sphinxattablestart
\centering
\begin{tabulary}{\linewidth}[t]{|T|T|}
\hline

\sphinxcode{type}
&
Type from {\hyperref[\detokenize{log:_CPPv2N6pessum7LogTypeE}]{\sphinxcrossref{\sphinxcode{LogType}}}} to convert to string
\\
\hline
\end{tabulary}
\par
\sphinxattableend\end{savenotes}

Determines that string corisponding to \sphinxcode{type} value.

\sphinxstylestrong{Return:} String corisponding to \sphinxcode{type} value.

\end{fulllineitems}



\subsection{SaveLog}
\label{\detokenize{log:savelog}}\index{pessum::SaveLog (C++ function)}

\begin{fulllineitems}
\phantomsection\label{\detokenize{log:_CPPv2N6pessum7SaveLogENSt6stringE}}%
\pysigstartmultiline
\pysiglinewithargsret{void \sphinxcode{}\sphinxbfcode{SaveLog}}{std::string \sphinxstyleemphasis{file}}{}%
\pysigstopmultiline~

\begin{savenotes}\sphinxattablestart
\centering
\begin{tabulary}{\linewidth}[t]{|T|T|}
\hline

\sphinxcode{file}
&
Path to file save log into
\\
\hline
\end{tabulary}
\par
\sphinxattableend\end{savenotes}

Saves the log entries from {\hyperref[\detokenize{log:_CPPv2N6pessum11global_logsE}]{\sphinxcrossref{\sphinxcode{global\_logs}}}} to specified file.

\end{fulllineitems}



\section{Variables}
\label{\detokenize{log:variables}}

\subsection{global\_logs}
\label{\detokenize{log:global-logs}}\index{pessum::global\_logs (C++ member)}

\begin{fulllineitems}
\phantomsection\label{\detokenize{log:_CPPv2N6pessum11global_logsE}}%
\pysigstartmultiline
\pysigline{\sphinxstrong{extern} std::vector\textless{}std::pair\textless{}int, std::string\textgreater{}\textgreater{} \sphinxcode{}\sphinxbfcode{global\_logs}}%
\pysigstopmultiline
All log calls are saved to this vector, and can be retrieved later with any
form of the {\color{red}\bfseries{}:function:{}`GetLog{}`} functions.

\end{fulllineitems}



\subsection{log\_handle\_full}
\label{\detokenize{log:log-handle-full}}\index{pessum::log\_handle\_full (C++ member)}

\begin{fulllineitems}
\phantomsection\label{\detokenize{log:_CPPv2N6pessum15log_handle_fullE}}%
\pysigstartmultiline
\pysiglinewithargsret{\sphinxstrong{extern} void (*\sphinxcode{}\sphinxbfcode{log\_handle\_full})}{std::pair\textless{}int, std::string\textgreater{}}{}%
\pysigstopmultiline
Pointer to function for handling log calls with full log information.
This function is called with every log entry added through {\color{red}\bfseries{}:function:{}`Log{}`}.

\end{fulllineitems}



\subsection{log\_handle}
\label{\detokenize{log:log-handle}}\index{pessum::log\_handle (C++ member)}

\begin{fulllineitems}
\phantomsection\label{\detokenize{log:_CPPv2N6pessum10log_handleE}}%
\pysigstartmultiline
\pysiglinewithargsret{\sphinxstrong{extern} void (*\sphinxcode{}\sphinxbfcode{log\_handle})}{std::string}{}%
\pysigstopmultiline
Pointer to function for handling logs with only formated string
This funtion is called with every log entry added through {\color{red}\bfseries{}:function:{}`Log{}`}.

\end{fulllineitems}




\renewcommand{\indexname}{Index}
\printindex
\end{document}